\documentclass[11pt]{article}
\usepackage{graphicx,wasysym, mdframed,xcolor,gensymb,verbatim}
\usepackage{color}
\usepackage{floatflt}
\usepackage[italian]{babel}
\usepackage[scaled=0.8]{FiraMono}
\definecolor{verdeoliva}{rgb}{0.3,0.3,0}
\definecolor{grigio}{rgb}{0.5,0.5,0.5}
\definecolor{blumarino}{rgb}{0.0,0,0.5}
\definecolor{panna}{rgb}{0.98,0.98,0.94}
\def\lstlistingname{Listato}
\lstset{%
  backgroundcolor=\color{panna},   % choose the background color; you must add \usepackage{color} or \usepackage{xcolor}; should come as last argument
% basicstyle=\footnotesize\ttfamily,
  basicstyle=\ttfamily,
%  belowskip=-0.8 \baselineskip,
% basicstyle=\footnotesize,        % the size of the fonts that are used for the code
  breakatwhitespace=false,         % sets if automatic breaks should only happen at whitespace
  breaklines=true,                 % sets automatic line breaking
  captionpos=b,                    % sets the caption-position to bottom
  commentstyle=\color{verdeoliva},    % comment style
% deletekeywords={},            % if you want to delete keywords from the given language
% escapeinside={\%*}{*)},          % if you want to add LaTeX within your code
% extendedchars=true,              % lets you use non-ASCII characters; for 8-bits encodings only, does not work with UTF-8
% firstnumber=1000,                % start line enumeration with line 1000
  frame=single,	                   % adds a frame around the code
  keepspaces=true,                 % keeps spaces in text, useful for keeping indentation of code (possibly needs columns=flexible)
  keywordstyle=\color{blue},       % keyword style
% language=Octave,                 % the language of the code
% morekeywords={*,},            % if you want to add more keywords to the set
  numbers=left,                    % where to put the line-numbers; possible values are (none, left, right)
  numbersep=5pt,                   % how far the line-numbers are from the code
  numberstyle=\tiny\color{grigio}, % the style that is used for the line-numbers
  rulecolor=\color{black},         % if not set, the frame-color may be changed on line-breaks within not-black text (e.g. comments (green here))
  showspaces=false,                % show spaces everywhere adding particular underscores; it overrides 'showstringspaces'
  showstringspaces=false,          % underline spaces within strings only
  showtabs=false,                  % show tabs within strings adding particular underscores
  stepnumber=1,                    % the step between two line-numbers. If it's 1, each line will be numbered
  stringstyle=\color{blumarino},   % string literal style
  tabsize=2,	                   % sets default tabsize to 2 spaces
  title=\lstname%                  % show the filename of files included with \lstinputlisting; also try caption instead of title
}

\def\cmu{\mbox{cm$^{-1}$}}
\def\half{\frac{1}{2}}

\voffset -2cm
\hoffset -2.5cm
%\marginparwidth 0cm
\textheight 22cm
\textwidth 17cm
%\oddsidemargin  0.2cm                                                                                         
%\evensidemargin 0.4cm                                                                                         
\parindent=0pt

\begin{document}
\pagestyle{empty}

\begin{center}
{\Large \bf  Laboratorio di Calcolo per Fisici, Sesta esercitazione\\[2mm]}
{\large Canale D-K, Docente: Lilia Boeri}
\end{center}
\vspace{4mm}

\begin{mdframed}[backgroundcolor=panna]
  Lo scopo della sesta esercitazione di laboratorio \`e di fare pratica con
le istruzioni di input/output da file e la scrittura modulare di programmi tramite funzioni.
  \end{mdframed}
%\vspace{1mm}
%
%



\hrule
\vspace{2mm}
La distribuzione binomiale, o distribuzione di Bernoulli, \`e una distribuzione di probabilit\`a
discreta che descrive la probabilit\`a che in $n$ esperimenti si verifichi $k$ volte un evento
con probabilit\`a individuale $p$.
La distribuzione \`e data da:
\[
P(k)=\frac{n!}{\nu! (n-k)!}p^{k}(1-p)^{n-k},
\]
(in questa formula $n!$ indica il {\em fattoriale\/} di n).
Trattandosi di una distribuzione discreta, si ha: $\sum_{k=0}^{N} P(k)=1$ ({\em condizione di normalizzazione\/}).

Esempi di processi descritti dalla distribuzione binomiale sono il lancio di una moneta ($p_{testa}=p_{croce}=\frac{1}{2}$), il lancio di un dado a 6 facce, l'estrazione di palline o bossoli da un'urna, l'estrazione (ripetuta) di una carta da un mazzo di carte, etc.

Consideriamo per esempio il lancio di un dado a 6 facce;
se definiamo come evento favorevole ({\em successo\/}) l'apparire di una determinata faccia, per esempio il numero 3,
la probabilit\`a $p$ che si verifichi il singolo evento sar\`a $p=\frac{1}{6}$; la distribuzione binomiale descriver\`a la probabilit\`a che in $n$ lanci successivi di un singolo dado a 6 facce appaia esattamente $k$ volte la faccia con il numero 3.

\vspace{2mm}
\hrule
\vspace{2mm}
\textbf{$\RHD$ Prima parte:}

\begin{enumerate}
\item Creare un programma \texttt{bernoulli.c} che calcoli e scriva su un file \texttt{bernoulli.dat} la
distribuzione di Bernoulli per valori arbitrari di $n$ e $p$. Il programma dovr\`a contenere almeno
una chiamata a una funzione esterna, che implementi la funzione fattoriale.
\item Il file di output deve contenere, su due colonne:
\\
\texttt{$0 \quad \quad P(0)$}
\\
\texttt{$k \quad \quad P(k)$}
\\
\texttt{$n \quad \quad P(n)$}

\item Dopo aver verificato che il programma funzioni correttamente e che la $P(k)$ soddisfi la condizione di
  normalizzazione, fissare $p=\frac{1}{6}$ e graficare con \texttt{python} la funzione $P(k)$ per i seguenti valori di $n$:
$n=2$; $n=4$; $n=10$; $n=20$.
Che cosa si nota all'aumentare di $n$? Perch\`e?
Salvate i dati usati per le figure su diversi file \texttt{bernoulli\_n.dat} e i grafici su file immagine
\texttt{bernoulli\_n.gif}.
\end{enumerate}

\hrule
\vspace{2mm}
\textbf{$\RHD$ Seconda parte:}
\vspace{2mm}

Creare un secondo programma \texttt{lanci.c} che simuli l'evento descritto dalla distribuzione di Bernoulli.
In particolare, il programma deve simulare una serie $N_{lanci}$ di lanci di $n$ dadi a 6 facce, e per
ciascuno di questi lanci registrare il numero di volte $k$ in cui compare la faccia numero 3 del dado.
L'istogramma delle occorrenze $k$, opportunamente rinormalizzato, tender\`a alla distribuzione di 
Bernoulli se il numero dei lanci $N_{lanci}$ \`e sufficientemente grande.
Il programma \texttt{lanci.c} deve:

\begin{enumerate}
\item Chiedere all'utente di inserire il numero $n$ di dadi lanciati per ciascun lancio e il numero di lanci $N_{lanci}$;
\item Chiamare una funzione esterna che simuli il lancio di un dado a 6 facce utilizzando un generatore di numeri casuali;
\item Costruire un istogramma che contenga il numero di successi $k$ per ciascun lancio.
\item Riscalare opportunamente l'istogramma ottenuto al punto precedente in modo che la {\em condizione di normalizzazione\/} della distribuzione sia soddisfatta.
\item Salvare su un file l'istogramma riscalato. Per il punto successivo, pu\`o essere utile aprire il file in modalit\`a aggiungi ({\em append\/}), in modo da non sovrascrivere i dati a ogni esecuzione del file.
\end{enumerate}


\hrule
\vspace{2mm}
\textbf{$\RHD$ Terza parte}
\vspace{2mm}

Confrontare (graficamente) i dati generati dai due programmi per gli stessi valori di $n$ e $k$.
Qual \`e il numero minimo di lanci da usare nel secondo programma per ottenere un campionamento ragionevole
della distribuzione di Bernoulli? Se si esegue il secondo programma pi\`u volte per uno stesso numero di lanci
$N_{lanci}$, come cambia la distribuzione? Che cosa si pu\`o fare per diminuire le {\em fluttuazioni\/} nel risultato?
Scrivete le vostre osservazioni e risposte su un file \texttt{risposte.c}.



\end{document}
