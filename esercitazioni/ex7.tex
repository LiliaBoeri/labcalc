%! TEX program = luatex
\documentclass[11pt]{article}
\usepackage{graphicx,wasysym, mdframed,xcolor,gensymb,verbatim}
\usepackage{color}
\usepackage{floatflt}
\usepackage[italian]{babel}
\newcommand{\persinfo}[3] {%
  \newcommand{\canale}{#1}
  \newcommand{\docente}{#2}
  \newcommand{\login}{#3}
}
\input{persinfo.tex}

\def\cmu{\mbox{cm$^{-1}$}}
\def\half{\frac{1}{2}}

\voffset -2cm
\hoffset -2.5cm
%\marginparwidth 0cm
\textheight 22cm
\textwidth 17cm
%\oddsidemargin  0.2cm                                                                                         
%\evensidemargin 0.4cm                                                                                         
\parindent 0pt

\begin{document}
\pagestyle{empty}

\begin{center}
{\Large \bf  Laboratorio di Calcolo per Fisici, Settima esercitazione\\[2mm]}
{\large Canale \canale, Docente: \docente}
\end{center}
\vspace{4mm}

\begin{mdframed}[backgroundcolor=gray!10]
  Lo scopo della settima esercitazione di laboratorio \`e di fare pratica con
gli argomenti appresi durante il corso in vista delle esercitazioni valutate;
l'argomento di questa esercitazione \`e la gestione di matrici $3 \times 3$.
  \end{mdframed}
%\vspace{1mm}
%
%



\hrule
\vspace{2mm}
Il determinante di una matrice quadrata $3 \times 3$ si pu`o calcolare usando la {\em regola di Sarrus}, che consiste in quanto segue: dalla matrice A, con$ N \! = \!  3$ righe 
e altrettante colonne, se ne ricava una $B$ con $N$ righe e $2N$ colonne in
cui la met`a destra di $B$ `e una copia esatta di $A$. Si calcolano quindi i prodotti $p$ degli elementi che si trovano
lungo tutte le $N$ diagonali che si possono costruire partendo dall’elemento in alto a sinistra e si sommano tra loro
algebricamente. Successivamente si calcolano gli $N$ prodotti degli elementi che si trovano lungo le $N$ diagonali che
si possono costruire in direzione opposta partendo dall’elemento in alto a destra di $B$. Questi prodotti si sommano
col segno cambiato a quanto ottenuto al passaggio precedente. Il risultato di questa somma `e il determinante della
matrice.

Consideriamo, per esempio, la matrice seguente:
\[
A = \left(\begin{array}{ccc}
5 &8& 6\\
3 &5 &9\\
7 & 9 & 9
\end{array}
\right)
\]
‚
Ricaviamo la matrice $B$ come
\[
B = \left(\begin{array}{cccccc}
5 & 8 & 6 &5 & 8 & 6\\
3 & 5 & 9 &3 &5  &9\\
7 & 9 & 9 &7 &9 & 9
\end{array}
\right)
\]
‚
Le possibili diagonali dall’elemento in alto a sinistra formano i seguenti prodotti: $(5 \times 5 \times 9)= 225$;
$(8 \times 9 \times 7)= 504$; 
e $(6\times3\times9)=162$. Quelle in direzione opposta sono invece: 
$(6\times5\times7)=210$, $(8\times3\times9)=216$ e $(5\times9\times9)=405$.
\\
Dati questi valori si calcola
$\det A = 225 + 504 +162 - 210 - 216 -405= 60$ .

\vspace{2mm}
\hrule
\vspace{2mm}
\textbf{$\RHD$ Prima parte:}

\begin{enumerate}
\item Creare un programma \texttt{sarrus.c} che calcoli il determinante di una matrice $3 \times 3$ 
utilizzando la formula di Sarrus.
\item Il programma dovr\`a offrire all'utente la possibilit\`a di inserire gli elementi da tastiera 
o di leggerli da file, e stampare il determinante su schermo.
\item Stampate  la matrice A sullo schermo in maniera leggibile.
\end{enumerate}

\hrule
\vspace{2mm}
\textbf{$\RHD$ Seconda parte:}
\vspace{2mm}
\begin{enumerate}
\item
Aggiungete al programma una funzione \texttt{genvec} che generi un vettore $\vec{x}$ con tre componenti {\em random} 
$\left(x_1,x_2,x_3\right)$.
\item Aggiungere una funzione \texttt{product} che calcoli il risultato del prodotto: 
$\vec{y}=A \vec{x}$ e lo salvi sul vettore $\vec{y}$.
\item Infine create una funzione \texttt{stampa}  perch\`e stampi su schermo in un formato elegante:
$\vec{y}=A \vec{x}$.
\end{enumerate}
{\bf N.B.:} Tutte le operazioni indicate sopra potrebbero essere eseguite direttamente nel \texttt{main}, ma lo scopo
di questa esercitazione \`e fare pratica con le funzioni che gestiscono vettori e array multidimensionali (paragrafo 7.3
del libro di testo). 
\\
\begin{mdframed}[backgroundcolor=gray!10]
{\em Un piccolo aiuto:} Per passare un array multidimensionale a una funzione \`e sempre necessario specificare la seconda  dimensione dell'array in maniera esplicita sia in fase di dichiarazione che di definizione della funzione;
il {\em prototipo} della funzione \texttt{prodotto} \`e perci\`o qualcosa del tipo:
\\
\texttt{
void product (float [][3], float *, float *);}.
\end{mdframed}



\end{document}
