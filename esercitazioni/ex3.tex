%! TEX program = luatex
\documentclass[11pt]{article}
\usepackage{textcomp}
\usepackage{graphicx,wasysym, mdframed,xcolor,gensymb,verbatim}
\usepackage{color}
\usepackage{floatflt}
\usepackage[italian]{babel}
\usepackage[scaled=0.8]{FiraMono}
\definecolor{verdeoliva}{rgb}{0.3,0.3,0}
\definecolor{grigio}{rgb}{0.5,0.5,0.5}
\definecolor{blumarino}{rgb}{0.0,0,0.5}
\definecolor{panna}{rgb}{0.98,0.98,0.94}
\def\lstlistingname{Listato}
\lstset{%
  backgroundcolor=\color{panna},   % choose the background color; you must add \usepackage{color} or \usepackage{xcolor}; should come as last argument
% basicstyle=\footnotesize\ttfamily,
  basicstyle=\ttfamily,
%  belowskip=-0.8 \baselineskip,
% basicstyle=\footnotesize,        % the size of the fonts that are used for the code
  breakatwhitespace=false,         % sets if automatic breaks should only happen at whitespace
  breaklines=true,                 % sets automatic line breaking
  captionpos=b,                    % sets the caption-position to bottom
  commentstyle=\color{verdeoliva},    % comment style
% deletekeywords={},            % if you want to delete keywords from the given language
% escapeinside={\%*}{*)},          % if you want to add LaTeX within your code
% extendedchars=true,              % lets you use non-ASCII characters; for 8-bits encodings only, does not work with UTF-8
% firstnumber=1000,                % start line enumeration with line 1000
  frame=single,	                   % adds a frame around the code
  keepspaces=true,                 % keeps spaces in text, useful for keeping indentation of code (possibly needs columns=flexible)
  keywordstyle=\color{blue},       % keyword style
% language=Octave,                 % the language of the code
% morekeywords={*,},            % if you want to add more keywords to the set
  numbers=left,                    % where to put the line-numbers; possible values are (none, left, right)
  numbersep=5pt,                   % how far the line-numbers are from the code
  numberstyle=\tiny\color{grigio}, % the style that is used for the line-numbers
  rulecolor=\color{black},         % if not set, the frame-color may be changed on line-breaks within not-black text (e.g. comments (green here))
  showspaces=false,                % show spaces everywhere adding particular underscores; it overrides 'showstringspaces'
  showstringspaces=false,          % underline spaces within strings only
  showtabs=false,                  % show tabs within strings adding particular underscores
  stepnumber=1,                    % the step between two line-numbers. If it's 1, each line will be numbered
  stringstyle=\color{blumarino},   % string literal style
  tabsize=2,	                   % sets default tabsize to 2 spaces
  title=\lstname%                  % show the filename of files included with \lstinputlisting; also try caption instead of title
}

\newcommand{\persinfo}[3] {%
  \newcommand{\canale}{#1}
  \newcommand{\docente}{#2}
  \newcommand{\login}{#3}
}
\input{persinfo.tex}

\def\cmu{\mbox{cm$^{-1}$}}
\def\half{\frac{1}{2}}

\voffset -2cm
\hoffset -2.5cm
%\marginparwidth 0cm
\textheight 22cm
\textwidth 17cm
%\oddsidemargin  0.2cm                                                                                         
%\evensidemargin 0.4cm                                                                                         
\parindent=0pt
\begin{document}
\pagestyle{empty}

\begin{center}
{\Large \bf  Laboratorio di Calcolo per Fisici, Terza esercitazione\\[2mm]}
{\large Canale \canale, Docente: \docente}
\end{center}
\vspace{4mm}

\begin{mdframed}[backgroundcolor=panna]
  Lo scopo della terza esercitazione di laboratorio \`e di fare pratica con
  le istruzioni di controllo di flusso \texttt{if...(then)...else}; \texttt{while...do};
\texttt{for...}, scrivendo un programma che localizza gli zeri di una funzione
all'interno di un intervallo dato.
  \end{mdframed}
%\vspace{1mm}
%
%

\hrule
\vspace{2mm}
\textbf{$\RHD$ Prima parte:}

Scrivere un programma \texttt{sinxx.c} che, utilizzando un opportuno ciclo iterativo,
calcoli il valore della funzione $f(x)=\frac{\sin(x)}{x}$ in una serie di
punti equispaziati lungo l'asse $x$, nell'intervallo $x \in \left[-15,15 \right]$.
Si noti che:
\begin{enumerate}
\item Creare una cartella  \texttt{EX3} che dovrà contenere il materiale di questa terza esercitazione.
\item  {\bf Il numero di punti} $N_{p}$ deve essere sufficiente a dar vita ad un grafico ragionevolmente continuo della funzione ($N_p \ge 100$);
\item Se la funzione viene calcolata nel punto $x=0$ il programma restituisce il valore NAN (\emph{not a number}), 
dal momento che si tratta di valutare un'espressione del tipo $\frac{0}{0}$. Il limite per $x \to 0$ di 
$\frac{\sin(x)}{x}$ \`e noto e vale: $\lim_{x \to 0} \frac{\sin(x)}{x} = 1$.
Inserire nel programma un'opportuna  istruzione \lstinline[language=c]!if! che corregga il risultato numerico con il valore analitico corretto in $x=0$.
\item Scrivere i risultati su un file \texttt{sinxx.dat} di $N_p$ righe e 2 colonne, nella forma: \texttt{x y}. Per compiere questa operazione potete o incollare manualmente l'output dello schermo su un file vuoto, o {\em ridirigere\/} l'output
del vostro programma su un file con il comando:
\\
\begin{lstlisting}[numbers=none,language=bash]
./programma.x > file_out
\end{lstlisting}
\item Creare con \texttt{python} un grafico della funzione $\frac{\sin(x)}{x}$
utilizzando i punti calcolati e salvarlo sul file \texttt{sinxx.png}. 
 Quanti zeri ci sono nell'intervallo $\left[ -15,15 \right]$? Dove sono?
\end{enumerate}

\hrule
\vspace{2mm}
\textbf{$\RHD$ Seconda parte:}

Scrivere un programma \texttt{zero.c} che calcoli automaticamente il numero degli zeri della funzione $f(x)=\frac{\sin(x)}{x}$ nell'intervallo $\left[a,b\right] = \left[-15,15 \right]$ mediante il seguente algoritmo,
basato sul teorema degli zeri di una funzione continua.
\\
({\em Attenzione, l'idea di base \`e simile a quella dell'algoritmo di bisezione, ma l'algoritmo \`e diverso!\/}).
\begin{enumerate}
\item L'intervallo di partenza viene diviso in $N$ intervalli contigui (attenzione: $N = N_p - 1$),
  tutti uguali, di estremi $x_L^{i}, x_R^{i}$, $i=0,\cdots,(N-1)$.
\item Per ciascun intervallo l'algoritmo controlla la presenza di eventuali 
zeri calcolando il valore di $s=f(x_L^{i})\cdot f(x_R^i)$. Se $s>0$,
non ci sono zeri; se $s \le 0$, c'\`e uno zero.
\item Alla fine dell'esecuzione, il programma restituisce il numero totale di zeri
trovati dall'algoritmo, N$_{zero}$.
\end{enumerate}

\newpage

\hrule
\vspace{2mm}
\textbf{$\RHD$ Terza parte (facoltativa):}
\begin{enumerate}
\item Far girare il programma \texttt{zero.c} con un numero variabile di intervalli 
$N \ge 1$ e creare un file che contenga $N, N_{zero}$.
\item Graficare l'andamento di $N, N_{zero}$: qual \`e il valore minimo di $N$
per cui vengono trovati tutti gli zeri?
\item Se il valore $N=3,4,5,6$ non fanno parte della vostra scelta iniziale, fate girare il programma anche per questi valori. Quanti zeri trovate? Perch\'e? 
{\em Scrivete le risposte a queste ultime due domande su un file risposte.txt
nella cartella \texttt{EX3}}.
\item Utilizzando diversi valori di $N$ abbastanza grandi da trovare tutti gli zeri,
fate stampare al vostro programma la vostra migliore stima del valore degli zeri ($x_L^i + x_R^i / 2$) con la relativa incertezza ($(x_R^i - x_L^i) / 2$).
\end{enumerate}



\begin{mdframed}[backgroundcolor=panna]
\textbf{$\RHD$ Attenzione!}
A volte, per via di un errore di programmazione, può succedere che un programma entri in un {\em loop infinito}.
Se questo succede, ci sono due modi di interrompere l'esecuzione:
\begin{itemize}
\item Se il programma è in esecuzione nella {\em shell\/}: premere la combinazione di tasti \texttt{CTRL+C}.
\item Se il programma è in esecuzione in modalit\`a {\em background\/}:
aprire una nuova shell. Lanciare il programma \texttt{top} (non in modalit\`a
 {\em background\/}). Individuare il PID ({\em Process Identifier\/}) del programma bloccato. Premere \texttt{q} per uscire da top.
Digitare della shell il comando: \texttt{kill -9 PID} (Enter).
\end{itemize}
\end{mdframed}

 

\end{document}
