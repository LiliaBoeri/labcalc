%! TEX program = luatex
\documentclass[11pt]{article}
\usepackage{textcomp}
\usepackage{graphicx,wasysym, mdframed,xcolor,gensymb,verbatim}
\usepackage{color}
\usepackage{floatflt}
\usepackage[italian]{babel}
%T1 garantisce la visualizzazione corretta del font  FiraMono usato per i codici
%con T1 i caratteri sono rappresentati a 8 bit e quindi si hanno a disposizione 256 glyphs
%la scelta migliore sare TU (unicode) ma questa è supportata solo da XeTeX e LuaTeX
\usepackage[scaled=0.8]{FiraMono}
\definecolor{verdeoliva}{rgb}{0.3,0.3,0}
\definecolor{grigio}{rgb}{0.5,0.5,0.5}
\definecolor{blumarino}{rgb}{0.0,0,0.5}
\definecolor{panna}{rgb}{0.98,0.98,0.94}
\def\lstlistingname{Listato}
\lstset{%
  backgroundcolor=\color{panna},   % choose the background color; you must add \usepackage{color} or \usepackage{xcolor}; should come as last argument
% basicstyle=\footnotesize\ttfamily,
  basicstyle=\ttfamily,
%  belowskip=-0.8 \baselineskip,
% basicstyle=\footnotesize,        % the size of the fonts that are used for the code
  breakatwhitespace=false,         % sets if automatic breaks should only happen at whitespace
  breaklines=true,                 % sets automatic line breaking
  captionpos=b,                    % sets the caption-position to bottom
  commentstyle=\color{verdeoliva},    % comment style
% deletekeywords={},            % if you want to delete keywords from the given language
% escapeinside={\%*}{*)},          % if you want to add LaTeX within your code
% extendedchars=true,              % lets you use non-ASCII characters; for 8-bits encodings only, does not work with UTF-8
% firstnumber=1000,                % start line enumeration with line 1000
  frame=single,	                   % adds a frame around the code
  keepspaces=true,                 % keeps spaces in text, useful for keeping indentation of code (possibly needs columns=flexible)
  keywordstyle=\color{blue},       % keyword style
% language=Octave,                 % the language of the code
% morekeywords={*,},            % if you want to add more keywords to the set
  numbers=left,                    % where to put the line-numbers; possible values are (none, left, right)
  numbersep=5pt,                   % how far the line-numbers are from the code
  numberstyle=\tiny\color{grigio}, % the style that is used for the line-numbers
  rulecolor=\color{black},         % if not set, the frame-color may be changed on line-breaks within not-black text (e.g. comments (green here))
  showspaces=false,                % show spaces everywhere adding particular underscores; it overrides 'showstringspaces'
  showstringspaces=false,          % underline spaces within strings only
  showtabs=false,                  % show tabs within strings adding particular underscores
  stepnumber=1,                    % the step between two line-numbers. If it's 1, each line will be numbered
  stringstyle=\color{blumarino},   % string literal style
  tabsize=2,	                   % sets default tabsize to 2 spaces
  title=\lstname%                  % show the filename of files included with \lstinputlisting; also try caption instead of title
}

\newcommand{\persinfo}[3] {%
  \newcommand{\canale}{#1}
  \newcommand{\docente}{#2}
  \newcommand{\login}{#3}
}
\input{persinfo.tex}

\newcommand{\voto}[1]{[\textbf{#1} punti]}
\def\cmu{\mbox{cm$^{-1}$}}
\def\half{\frac{1}{2}}

\voffset -2cm
\hoffset -2.5cm
%\marginparwidth 0cm
\textheight 22cm
\textwidth 17cm
%\oddsidemargin  0.2cm                                                                                         
%\evensidemargin 0.4cm                                                                                         
\parindent=0pt

\begin{document}
\pagestyle{empty}

\begin{center}
{\Large \bf  Laboratorio di Calcolo per Fisici, Prima esercitazione\\[2mm]}
{\large Canale \canale, Docente: \docente}
\end{center}
\vspace{4mm}

\begin{mdframed}[backgroundcolor=panna]
  Lo scopo della prima esercitazione di laboratorio \`e di introdurre gli strumenti di base che verranno usati nel corso delle successive esercitazioni:
  la {\em shell}, l'editor di testo, il compilatore {\em gcc}, e {\em python\/} (tramite le librerie
  {\em matplotlib\/} e {\em numpy\/}) per la grafica.
\\
 % Sulla pagina web del corso
 % ({\em https://lboeri.wordpress.com/teaching/labcalc/ex/})
 % sono disponibili dei tutorial pi\`u avanzati su ciascuno di questi argomenti.
\end{mdframed}
%\vspace{1mm}
%
%

\hrule
\vspace{2mm}
\textbf{$\RHD$ Prima parte (obbligatoria)} 
\begin{enumerate}
\item Effettuare il {\em login\/} sulla propria macchina {\em Unix\/} utilizzando lo {\em userid\/} \texttt{\login}, 
dove $xx$ \`e il numero del gruppo a cui siete stati assegnati.
\item Aprire una finestra di {\em terminale}.
\item Creare una cartella  \texttt{EX1} che conterrà il materiale della prima esercitazione.
\item Nella cartella \texttt{EX1} aprire con l'editor di testo il file \texttt{temp.c}, e digitare il listato sottostante. Salvare il contenuto del file. {\em Suggerimento:} Per entrare nella cartella \texttt{EX1} usare il comando linux \texttt{cd}.

\item Compilare il programma in c digitando sul terminale:
  \texttt{gcc temp.c -o temp.x}
\item Eseguire il file {\em temp.x\/} digitando \texttt{./temp.x}
\item Inserire i dati richiesti dal programma; il programma \`e un semplice convertitore di temperature da gradi Celsius a gradi Fahrenheit.
\item Sempre nella cartella \texttt{EX1} aprire con l'editor di testo il file \texttt{ex1\_2.py} e digitare il listato
  contenuto nella prossima pagina rispettando accuratamente l'indentazione. Il file \texttt{ex1\_2.py} contiene un
  semplice script in linguaggio \texttt{python} per creare un plot (grafico) nel piano cartesiano, utilizzando i dati contenuti in un file con due colonne $x,y$. 
\end{enumerate}
\begin{lstlisting}[caption={Programma \texttt{temp.c}},language=c]
#include <stdio.h>

int main()
{
  double tc, tf, conv, offset;
  conv = 5./9.;
  offset = 32.;
  printf("Inserisci la temperatura in gradi Celsius \n");
  scanf("%lf", &tc);
  tf = tc/conv + offset;
  printf("La temperatura in gradi Fahrenheit vale %5.2f gradi\n",tf);
}
\end{lstlisting}

\newpage
\begin{lstlisting}[caption={Programma {\it python} \texttt{ex1\_2.py}},language=Python]
#!/usr/bin/env python3
import matplotlib.pyplot as plt
import numpy as np
plt.title('Un primo plot con Python')
x, y = np.loadtxt('temp.dat', unpack=True)
plt.plot(x ,y, 'x', label='Temperature caricate da file')
plt.xlim((-10,50)) # intervallo lungo asse x
plt.ylim((10,125)) # intervallo lungo asse y
plt.show()
\end{lstlisting}
\vspace{2mm}
\hrule
\vspace{2mm}
\textbf{$\RHD$ Seconda parte (obbligatoria)} 
%
\begin{enumerate}
\item Eseguire il programma \texttt{temp.x} quattro o più volte, con valori di input diversi, e creare un file di testo chiamato \texttt{temp.dat} con due colonne,
  che contenga i valori delle temperature in Celsius (Tc) e
  in Fahrenheit (Tf), cioè i valori di input e output del programma \texttt{temp.c}.
%\item 
  %Aprire il programma di grafica \texttt{gnuplot} digitando nella shell il
  %comando \texttt{gnuplot}.
\item Dal terminale graficare i dati contenuti nel file \texttt{temp.dat} con il comando:\\
  \begin{lstlisting}[language=bash,numbers=none]
    python3 ./ex1_2.py
\end{lstlisting}
\item Aggiungere le legende all'asse $x$ e $y$ aprendo il file \texttt{ex1\_2.py} con l'editor di testo e aggiungendo prima di \texttt{plt.show()} i comandi:
\\
\texttt{plt.xlabel('Tc')}
\\
\texttt{plt.ylabel('Tf')}
         % {\em Ogni volta che si d\`a un nuovo comando a gnuplot bisogna ricordarsi di fare il refresh dello schermo con il comando \texttt{replot}.}
 \item Eseguire nuovamente lo script.
 \end{enumerate}


\hrule
\vspace{2mm}\textbf{$\RHD$ Terza parte (facoltativa)\\}
%
Con {\it Python\/} e le librerie {\it matplotlib\/} \`e possibile graficare non solo dati contenuti in un file esterno, ma anche funzioni definite dall'utente. Per esempio, 
per plottare la funzione $y=x$ \`e sufficiente aggiungere prima di \texttt{plt.title}: 
\begin{lstlisting}[language=python]
x = np.linspace(-10, 50, 100)
y = x
plt.plot(x, y, label='retta y=x')
\end{lstlisting}
In questo modo viene aggiunto il plot della retta $y=x$ nell'intervallo $[-10,50]$. 

\begin{enumerate}
  \item Utilizzando le funzionalit\`a in {\it python\/} appena descritte, disegnare la retta che interpola i dati generati dal programma \texttt{temp.x}.
\item 
%Utilizzando l'{\em help} di {\em gnuplot}, che si invoca con il comando \texttt{help},
Inserire una legenda nel grafico per la retta interpolante aggiungendo nel programma \textit{python} prima di 
\lstinline[language=python]!plt.show()! 
il comando \lstinline[language=python]!plt.legend()! e riscalare gli assi $x$ e $y$ in modo che vadano dalla più piccola alla più grande delle temperature scelte. {\em Suggerimento}: Per modificare il range degli assi $x$ e $y$ modificare gli argomenti di \texttt{plt.xlim()} e \texttt{plt.ylim()}.
\end{enumerate}  
\end{document}
